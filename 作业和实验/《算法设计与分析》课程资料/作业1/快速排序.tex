% !TeX root = ./homework.tex
\section{快速排序}
\subsection{问题描述}
\subsubsection*{Description}

给定一维\code{int}型数组\code{a[0,1,...,n-1]},使用快速排序方法,对其进行从小到大排序,请输出递归过程中自顶自下第二层的划分结果,其中最顶层为第一层,即最终的排序结果层。

划分时请用第1个元素作为划分基准,并使用课件上的方法进行一次扫描实现划分。

\subsubsection*{Input}

输入第1行有一个\code{int}型正整数\code{m}(\code{m}<100),表示有\code{m}行输入。
每行输入的第一个数为\code{int}型正整数\code{n}(8<\code{n}<1000),后面接着输入\code{n}个\code{int}型整数。

\subsubsection*{Output}

对每组数据, 输出自顶自下第二层的划分结果。

\subsubsection*{Sample Input}

\code{2}

\code{11 6 3 7 8 5 1 4 2 4 9 10}

\code{12 6 3 7 8 4 5 1 11 2 4 9 10}

\subsubsection*{Sample Output}

\code{2 3 1 4 4 5 6 7 8 9 10}

\code{2 3 1 4 4 5 6 10 8 7 9 11}

\subsection{算法思路}

\begin{enumerate}
    \item 划分:以数组中的第一个元素为基准值,从数组中的第二个元素开始扫描,比基准值小的放右边,比基准值大的放左边。
    \item 处理:
    \begin{itemize}
        \item 初始时基准坐标\code{p=}数组开头位置;
        \item \code{i}从数组第二个元素开始遍历,若\code{i}位置的值大于基准值,则与\code{p}位置后一位的值交换,并令\code{p}自增;
        \item 最后令数组开头的值与\code{p}位置的值交换。
    \end{itemize}
    \item 递归:使用同样的方法递归地处理左边和右边的子数组。
    \item 输出第二层结果:设置一个层标记\code{l=1},每次递归都将此值加1后作为参数传入,递归函数内若检测到\code{l>2}则退出并输出结果。
\end{enumerate}

\subsection{算法伪代码}
见算法\ref{alg:qs}。
\begin{algorithm}[t]
\caption{快速排序算法伪代码}\label{alg:qs}
\SetKwProg{Fn}{Function}{ begin}{end}
\Fn{QuickSort($S$, $n$)}{
    \KwIn{未排序的数组$S$、数组长度$n$、递归层数$l$}
    \KwOut{排好序的数组$S$}
    \If{$(l\leq 2)\wedge (n>1)$}{
        $p = 0$\;
        \For{$i\in\{1,2,3,...,n\}$}{
            \If{$S_0>S_{i}$}{
                交换$S_{p+1}$和$S_{i}$\;
                $p$自增1\;
            }
        }
        交换$S_0$和$S_p$\;
        QuickSort($S$, $p$, $l + 1$)\;
        QuickSort($S'=\{S_i|i\in[p+1,n-1]\}$, $n - (p + 1)$, $l + 1$)\;
    }
    \Return
}
\end{algorithm}