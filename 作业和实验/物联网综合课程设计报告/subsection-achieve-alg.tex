\subsubsection{模型资源说明}
\begin{itemize}
	\item cascade.xml:检测模型 - 目前效果最好的cascade检测模型
	\item cascade\_lbp.xml:召回率效果较好,但其错检太多
	\item char\_chi\_sim.h5:Keras模型-可识别34类数字和大写英文字 使用14W样本训练
	\item char\_rec.h5:Keras模型-可识别34类数字和大写英文字 使用7W样本训练
	\item ocr\_plate\_all\_w\_rnn\_2.h5:基于CNN的序列模型
	\item ocr\_plate\_all\_gru.h5:基于GRU的序列模型从OCR模型修改,效果目前最好但速度较慢,需要20ms。
	\item plate\_type.h5:用于车牌颜色判断的模型
	\item model12.h5:左右边界回归模型
\end{itemize}

\subsubsection{Python依赖}
\begin{itemize}
	\item Keras (>2.0.0)
	\item Theano(>0.9) or Tensorflow(>1.1.x)
	\item Numpy (>1.10)
	\item Scipy (0.19.1)
	\item OpenCV(>3.0)
	\item Scikit-image (0.13.0)
	\item PIL
\end{itemize}

\subsubsection{参数说明}
\begin{itemize}
	\item detect\_path: 被检测图片的路径,default = None;
	\item cascade\_model\_path: 用于object detection的模型文件路径default = model/ cascade.xml;
	\item mapping\_vertical\_model\_path: 用左右边界回归模型文件路径default = model/model12.h5;
	\item ocr\_plate\_model\_path: 用于检测车牌中的文字default = model/ ocr\_plate\_all\_gru.h5;
	\item result\_save\_folder\_path: 识别结果图片存储路径folder (None表示不存储)default = None。
\end{itemize}

\subsubsection{源码解析}
\begin{enumerate}[label=\arabic*、]
	\item {\bf 入口文件 demo.py(附录\ref{apdx:demo.py})}

	      opencv2的imread函数导入图片, 返回的是Mat类型。HyperLPRLite.py中的LPR类构造函数导入model, 参数就是训练好的三个模型文件,分别是:
	      \begin{itemize}
		      \item model/cascade.xml;
		      \item model/model12.h5;
		      \item model/ocr\_plate\_all\_gru.h5
	      \end{itemize}

	\item {\bf HyperLPRLite.py(附录\ref{apdx:HyperLPRLite.py})}
	
	      参数 model\_detection 就是文件 model/cascade.xml。用到了 opencv2的CascadeClassifier()函数 cv2.CascadeClassifier(),参数输入.xml或者.yaml文件加载模型。

	\item {\bf 基于Haar特征的级联分类器用于物体检测的模型}
	
	      model.SImpleRecognizePlateByE2E()函数(附录\ref{apdx:SImpleRecognizePlateByE2E})
		  输入为一个Mat类型的图片,输出为识别的车牌字符串以及可信度confidence,该函数定义在 HyperLPRLite.py(附录\ref{apdx:HyperLPRLite.py})。其中detectPlateRough()函数(附录\ref{apdx:detectPlateRough})是返回图像中所有车牌的边框在图片中的bbox,返回值是一个表示车牌区域坐标边框的list。for循环中,对于每个识别出来的车牌用到filemappingVertical()(附录\ref{apdx:filemappingVertical})
		  
	      输入参数:
	      \begin{itemize}
		      \item image\_gray: 一个rgb图像,Mat类型;
		      \item resize\_h: 重新设定的图像大小;
		      \item top\_bottom\_padding\_rate: 表示要裁剪掉图片的上下部占比。
	      \end{itemize}

	\item {\bf detectPlateRough()函数(附录\ref{apdx:detectPlateRough})}
	
	      这个函数实现的功能如下:
	      \begin{enumerate}
		      \item resize图像大小:cv2.resize函数;
		      \item 裁剪图片:输入的top\_bottom\_padding\_rate如果是0.1,那么上面裁剪掉0.1*height,下面也裁剪掉0.1*height;
		      \item 将图像从rgb转化为灰度 cv2.cvtColor函数,cv2.COLOR\_RGB2GRAY;
		      \item 根据前面的cv2.CascadeClassifier()物体检测模型(3),输入image\_gray灰度图像,边框可识别的最小size,最大size,输出得到车牌在图像中的offset,也就是边框左上角坐标( x, y )以及边框高度( h )和宽度( w );
		      \item 对得到的车牌边框的bbox进行扩大,也就是宽度左右各扩大0.14倍,高度上下各扩大0.15倍;
		      \item 返回图片中所有识别出来的车牌边框bbox,这个list作为返回结果。
	      \end{enumerate}


	\item {\bf filemappingVertical函数(附录\ref{apdx:filemappingVertical})}
	
	      输入参数:裁剪的车牌区域图像(Mat类型),rect也是裁剪的车牌部分的图像(Mat类型)

	      实现功能:
	      \begin{enumerate}
		      \item 将原来车牌图像resize大小:66*16*3;
		      \item 将原来灰度图颜色通道[0, 255]转化为float类型[0,1];
		      \item 将输入66*16(float),输入进模型进行测试。
	      \end{enumerate}

	\item {\bf ModelFineMapping模型(附录\ref{apdx:ModelFineMapping})}
	
	      model\_finemapping()函数(附录\ref{apdx:model_finemapping})实现keras网络模型对车牌的左右边界进行回归;通过modelFineMapping.loadweights()函数加载模型文件;通过modelFineMapping.predict输出网络结果。

	      输入:16*66*3 tensor

	      输出:长度为2的tensor

	\item {\bf ocr识别(附录\ref{apdx:ocr})}
	
	      对于每个车牌区域的for循环中,经过fineMappingVertical处理后输入到recognizeOne函数(见附录\ref{apdx:recognizeOne})进行ocr识别。

	\item {\bf modelSecRec模型}
	
		  基于GRU的序列模型从OCR模型中修改的网络模型。
		  
		  model\_sec\_rec函数(附录\ref{apdx:model_sec_rec})输入model\_path为模型weights文件路径;ocr部分的网络模型(keras模型)。
		  
		  输入层:164*48*3的tensor
		  
	      输出层:长度为7 的tensor,类别有len(chars)+1种(附录\ref{apdx:chars})

\end{enumerate}
