\subsection{红外传感器工作原理}
红外传感器是能将红外辐射能转换为电能的光敏器件。

红外线是太阳光线中众多不可见光线中的一种,与所有电磁波一样,具有反射、折射、散射、干涉、吸收等性质。利用红外线可以被物体反射这一性质,在一定范围内,反射管发射出一定频率的红外线,如果检测方向上没有障碍物,发射出去的红外线,因为传播的距离越来越远而逐渐减弱,最后消失;当检测方向存在障碍物时,发射出去的红外线反射回来被接收管接收,进而改变红外传感器输出电平,由此判断检测方向上是否存在障碍物。

\subsection{步进电机工作原理}
在额定负载一定时,步进电机停止的位置取决于输入脉冲信号的数量,而步进电机运行的速度取决于脉冲信号的频率。在实际使用中,负载的变化不会对两项参数造成很大影响。在输入端给与适当的脉冲信号时,转子会以一定的方式旋转;输入端无脉冲信号时,转子会保持当前的位置不动。步进电机工作的基本原理如下:
\begin{enumerate}
	\item 换相顺序的控制

		  步进电机上电后,工作相序的控制取决于脉冲信号的分配。以最基本的四相八拍为例,要求的工作顺序为A-AB-B-BC-C-CD-D-DA,输入端的脉冲就会以A、B、C、D这种顺序进行相序的通断设置。还有四相四拍运行方式,即AB-BC-CD-DA-AB;
		  
	\item 步进电机转向的改变

	      如果要求电机正向转动,输入端就要按照正向的顺序为电机通电;同理如果要求电机反向转动,输入端就要按照相反的顺序来为电机通电;

	\item 步进电机转动速度的改变

	      每当输入端接收到一个脉冲信号,步进电机的转子就会转动一次,产生一定的转动角度。所以输入端两次脉冲的时间间隔决定步进电机的转动速度,两次脉冲发送间隔越长,对应的步进电机转动速度就越慢。如果需要对步进电机转动速度加以控制,就要对主控模块在单位时间内发出的脉冲个数进行控制。
	      本系统采用28BYJ-48四相五线式步进电机。电机公共端连接电源正极,电机剩余的四根控制线顺次与电源的接地端(此处可理解为负极)连接。接线时可以看到,每当电机控制线与电源地线接触一次,步进电机就会旋转一定角度。
\end{enumerate}

\subsection{中文车牌识别原理}
中文车牌识别,经过近二十年的发展,在特定场景下,已经具备了相对成熟的解决方案。如停车场卡口,小区入口等。车牌识别技术是现代智能交通系统重要组成部分,其应用十分广泛。它以计算机视觉处理、数字图像处理、模式识别等技术为基础,对摄像机所拍摄的车辆图像或者视频图像进行处理分析,得到每辆车的车牌号码,从而完成识别过程。车牌识别在高速公路车辆管理中得到广泛应用,如高速收费,交通违章检测等。在停车场管理中,车牌识别技术也是识别车辆身份的主要手段。

本项目中我们借鉴学习并改进使用了Github上一个比较优秀的开源方案HyperLPR。HyperLPR是一个使用深度学习针对对中文车牌识别的实现,与较为流行的开源的EasyPR相比,它的检测速度和鲁棒性和多场景的适应性更好,且可以识别多种中文车牌,包括白牌,新能源车牌,使馆车牌,教练车牌,武警车牌等。HyperLPR的优点主要有以下几条:
\begin{itemize}
	\item 速度快:720p,单核 Intel 2.2G CPU 平均识别时间低于100ms;
	\item 基于端到端的车牌识别无需进行字符分割;
	\item 识别率高:仅仅针对车牌ROI在EasyPR数据集上,0-error达到95.2\%,1-error识别率达到 97.4\% (指在定位成功后的车牌识别率);
	\item 轻量:总代码量不超1000行。
\end{itemize}

\subsection{长轮询通信原理}
长轮询(Long Polling)是一种HTTP实现的“服务器推”的技术,它弥补了HTTP简单的请求应答模式无法保证数据即时交互的不足,极大地增强了程序的实时性和交互性。长轮询一般应用与WebIM、ChatRoom和一些需要及时交互的网站应用中。其真实案例有:WebQQ、Hi网页版、Facebook IM等。

在长轮询中,客户端向服务器发送轮询请求,服务器接到请求后保持连接,直到有新消息才返回响应信息并关闭连接,客户端处理完响应信息后再向服务器发送新的轮询请求。与传统的HTTP请求应答模式不同,长轮询方式在无消息的情况下不会频繁的请求,且在服务器端有新消息回应时能立即接收,在降低了网页即时通信带宽消耗的同时极大地提升了消息传输的性能。

\subsection{微服务架构原理}
微服务架构(Microservice Architecture)是一种架构概念,旨在通过将功能分解到各个离散的服务中以实现对解决方案的解耦。微服务架构把一个大型的单个应用程序和服务拆分为数个甚至数十个支持微服务,它可扩展单个组件而不是整个的应用程序堆栈,从而满足服务等级协议。这些支持微服务可独立地进行开发、管理和迭代。在分散的组件中使用云架构和平台式部署、管理和服务功能,使产品交付变得更加简单。

Docker是针对微服务架构开发最具代表性的应用之一。它是一个基于PaaS (Platform as a Service:平台即服务)思想的开源应用容器引擎,让开发者可以打包他们的微服务应用以及相关的依赖包到一个可移植的容器中,然后使用虚拟化方法部署到任何安装了Docker的物理服务器上,容器是完全使用沙箱机制,相互之间不会有任何依赖接口。

基于Docker的软件交付运行环境如同海运,其中虚拟化的操作系统镜像(image)如同一个货轮,在其中运行的微服务如同一个个集装箱,用户可以通过标准化手段自由组装运行环境,集装箱的内容可以由用户自定义,也可以由专业人员制造。这样,交付软件就被分解为交付一系列标准化微服务集合,从而更有效地实现微服务架构开发。
