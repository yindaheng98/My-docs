\documentclass{article}
\usepackage{metalogo} % for the logo of XeLaTeX
\usepackage{xeCJK}

\begin{document}
	
\title{智能制造概论结业报告}
  \section{前言}
  随着物联网、大数据和人工智能等新兴信息技术的发展,全球工业革命正逐步临近。2015年李克强总理在《政府工作报告》中提到了“中国制造2025”, 指出:要实现中国制造, 就需要坚持创新驱动、智能转型、强化基础。把握新的发展机遇、实现工业转型,必将有利于中国从制造大国向制造强国的转变;而智能制造作为新一代工业革命的重要实践模式,已经引发了行业的广泛关注。本文将从智能制造的层次入手,对智能制造这一新型制造模式做一个简要的介绍。
  \section{智能制造的层次}
  \subsection{数字化工厂}
  德国工程师协会对数字化工厂的定义是\cite{c2}:数字化工厂(DF)是由数字化模型、方法和工具构成的综合网络,包含仿真和3D可视化,并通过连续的没有中断的数据管理集成在一起。相对于通常意义上的工厂对产品和生产流程的集成,数字化工厂除产品和生产流程外还集成了工厂模型以及和生产流程相关的数据库,并通过先进的数据可视化技术、虚拟仿真技术以及数据管理技术提高产品质量和生产过程的动态性能。
在国内,对于数字化工厂接受度最高的定义是\cite{c3}(百度):以产品全生命周期的相关数据为基础,在计算机虚拟环境中,对整个生产过程进行仿真、评估和优化,并进一步扩展到整个产品生命周期的新型生产组织方式。数字化工厂是传统的自动化制造技术与先进的计算机仿真技术的一种集成,其使用计算机仿真技术代替人工在产品设计和产品制造之间进行沟通交流,从而提高了产品从设计到制造这一过程的效率。
  \subsection{智能工厂}
  智能工厂是在数字化工厂的基础上, 利用物联网技术和监控技术加强信息管理服务, 最终达到最优生产、无人干预、效益最佳、动态平衡的目标。\cite{c4}。除虚拟仿真外,智能工厂还集成了众多现代化的信息监控和信息管理技术,以提高生产过程的可控性,减少人为干预,合理安排生产计划;同时对智能系统进行初步集成,构建高效且人性化的生产环境。
  在智能工厂这一层次,生产过程已经具有了一定的自主能力,可以进行信息的采集分析、生产过程的推理预测、以及生产计划的规划调整。工厂中的各组成部分可自行组成最优结构,具备自协调、自组织能力,并且可以进行生产环节的修改和扩充。因此,智能工厂将人与机器的关系从人控制机器转变为人与机器互相协调,使机器脱离了不够精准而且极易出错的人工控制,极大地提高了产品质量和生产效率。
  \subsection{智能制造系统}
  智能制造系统是一种由智能装备、智能控制和智能信息共同组成的人机一体化制造系统,它集合了人工智能、柔性制造、虚拟制造、系统控制、网络集成、信息处理等学科和技术的发展成果。\cite{c5}智能制造系统的实质就是信息物理系统CPS(Cyber physical system),它是通过人机交互接口实现和物理进程的交互,使物理系统具有计算、通信、精确控制、远程协作和自治功能,更加深化了信息技术和生产技术的融合。
智能制造在智能工厂系统的基础上进行进一步的集成,通过各种技术将智能工厂整合进入更大规模的、集需求、生产、制造、运输、销售、维护等多个环节于一体的智能制造系统中。
因此,智能制造系统是与制造相关的多种系统进行集合,并使各部分协调工作。智能制造系统使智能工厂与其他系统更加协调地工作,在内外两方面提升了工业制造的智能化与人性化。
  \section{总结}
  当前, 我国制造业在世界制造产业链中处于最低端, 生产企业单位加工产品利润少, 产品附加值不高, 而智能制造系统将最大限度地降低产品生产成本, 使加工后的成品价格无限接近生产成本, 并使大规模定制化生产成为可能, 极大提升生产效率, 中国制造业需要智能制造系统的生产效率。\cite{c6}
  \begin{thebibliography}{3}
  	\bibitem{c2}https://wenku.baidu.com/view/dd45825f26284b73f242336c1eb91a37f11132de.html
  	\bibitem{c3}https://baike.baidu.com/item/智能化工厂/8480700?fr=aladdin
  	\bibitem{c4}马孟模.流程工业智能工厂建设技术应用探究[J].工业控制计算机,2017,30(03):53-54+57.
  	\bibitem{c5}刘昭斌,刘文芝,顾才东,张玉成.基于智能制造系统的物联网3D监控[J].实验技术与管理,2015,32(02):89-93.
  	\bibitem{c6}陈丽娟.我国智能制造产业发展模式探究——基于工业4.0时代[J].技术经济与管理研究,2018(03):109-113.
  \end{thebibliography}
\end{document}