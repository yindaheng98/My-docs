% !TeX root = ./homework.tex
\section{最低票价}
\subsection{问题描述}
\subsubsection*{Description}
在一个火车旅行很受欢迎的国度,你提前一年计划了一些火车旅行。在接下来的一年里,你要旅行的日子将以一个名为\code{days}的数组给出。每一项是一个从\code{1}到\code{365}的整数。

火车票有三种不同的销售方式:

\begin{itemize}
    \item 一张为期一天的通行证售价为\code{costs[0]}美元;
    \item 一张为期七天的通行证售价为\code{costs[1]}美元;
    \item 一张为期三十天的通行证售价为\code{costs[2]}美元。
\end{itemize}

通行证允许数天无限制的旅行。例如,如果我们在第 2 天获得一张为期 7 天的通行证,那么我们可以连着旅行 7 天:第 2 天、第 3 天、第 4 天、第 5 天、第 6 天、第 7 天和第 8 天。

返回你想要完成在给定的列表\code{days}中列出的每一天的旅行所需要的最低消费。

\subsubsection*{Input}
第一行输入\code{nums}表示有\code{nums}组测试

对每组测试用例

第一行输入\code{m}

第二行输入具有\code{m}个元素的\code{days}数组,\code{days[i]}表示你将在\code{days[i]}这天旅行

第三行输入具有3个元素的\code{costs}数组,具体释义见Description

\subsubsection*{Output}

对每组测试数据,输出你想要完成在给定的\code{days}数组 中列出的每一天的旅行所需要的最低消费。

\subsubsection*{Sample Input}

\code{2}

\code{6}

\code{1 4 6 7 8 20}

\code{2 7 15}

\code{12}

\code{1 2 3 4 5 6 7 8 9 10 30 31}

\code{2 7 15}

\subsubsection*{Sample Output}

\code{11}

\code{17}

\subsubsection*{提示}

\code{1 $\leq$ days.length $\leq$ 365}

\code{1 $\leq$ days[i] $\leq$ 365}

\code{days}按顺序严格递增

\code{costs.length $\equiv$ 3}

\code{1 $\leq$ costs[i] $\leq$ 1000}

\subsection{算法思路}

设要查找的两个正序数组分别为$\mathbb M$和$\mathbb N$,其组合成的正序数组为$\mathbb U$。满足:
$$
\begin{aligned}
    &(\forall M_i\in\mathbb M)(M_{i+1}\geq M_i)\wedge(\forall N_i\in\mathbb N)(N_{i+1}\geq N_i)\wedge\\
    &\mathbb U=\{U_i|U_i\in\mathbb M\cup\mathbb N\}\wedge(\forall U_i\in\mathbb U)(U_{i+1}\geq U_i)\wedge\\
    &k<|\mathbb M|\wedge k<|\mathbb N|
\end{aligned}
$$

则有如下定理:

$$
M_k<N_k\Rightarrow M_k<U_{2k}
$$

证明:反证法。

$$
\begin{aligned}
    &M_k<N_k\wedge M_k\geq U_{2k}\\
    \Leftrightarrow& U_{2k}\leq M_k<N_k\\
    \Rightarrow& U_{2k}\in\{M_i\in\mathbb{M}|1\leq i\leq k\}\cup\{N_i\in\mathbb{N}|1\leq i\leq k-1\}\\
    \Rightarrow &|\{U_i\in\mathbb{U}|1\leq i\leq 2k\}|<|\{M_i\in\mathbb{M}|1\leq i\leq k\}\cup\{N_i\in\mathbb{N}|1\leq i\leq k-1\}|\\
    \Rightarrow &2k\leq 2k-1\Rightarrow false\\
\end{aligned}
$$

$\therefore M_k<N_k\rightarrow M_k<U_{2k}$为真命题。

查找两个正序数组$\mathbb M$和$\mathbb N$的中位数的算法可以等价为一个查找其组合成的正序数组$\mathbb U$中第$2k$大数$U_{2k}$的算法:
\begin{enumerate}
    \item 若两数组长度和为偶数,则$2k=(|\mathbb M|+|\mathbb N|)/2$,查找$U_{2k}$和$U_{2k+1}$取平均值;
    \item 若两数组长度和为奇数,则$2k=(|\mathbb M|+|\mathbb N|+1)/2$,查找$U_{2k}$。
\end{enumerate}

而根据前述定理$M_k<N_k\Rightarrow M_k<U_{2k}$,若$M_k<N_k$,则$U_{2k}$必然不在$\{M_i\in\mathbb{M}|1\leq i\leq k\}$中,因此可以直接舍弃该部分,在$\{M_i\in\mathbb{M}|k+1\leq i\leq |\mathbb M|\}$和$\mathbb N$中查找第$k$大的元素即可。显然,对于$M_k>N_k$时类似的情况也成立。此过程可以递归进行,直到$M_k=N_k$或某一轮的$\mathbb{M}$或$\mathbb{N}$为空。算法的时间复杂度为$O(log|\mathbb M|+log|\mathbb N|)$。

\subsection{算法伪代码}
见算法\ref{alg:fk}和算法\ref{alg:center}。
\begin{algorithm}[htbp]
\caption{在两个正序数组中找第$k$大数}\label{alg:fk}
\SetKwProg{Fn}{Function}{ begin}{end}
\Fn{FindK($\mathbb M$, $\mathbb N$, $k$)}{
    \KwIn{数组$\mathbb M$和$\mathbb N$、整数$k$}
    \KwOut{数组$\mathbb M$和$\mathbb N$中第$k$大的数}
    \lIf{$\mathbb M=\emptyset$}{\Return $N_k$}
    \lIf{$\mathbb N=\emptyset$}{\Return $M_k$}
    \lIf{$k=1$}{\Return $min(M_1,N_1)$}
    $d=\left\lfloor\frac{k}{2}\right\rfloor$\;
    \If{$d>|\mathbb M|$}{
        \eIf(){$M_{|\mathbb M|}\leq N_d$}{\Return $FindK(\emptyset, \mathbb N, k-|\mathbb M|)$\;}{\Return $FindK(\mathbb M, \{N_i\in\mathbb N|i>d\}, k-d)$\;}
    }
    \If{$d>|\mathbb N|$}{
        \eIf{$N_{|\mathbb N|}\leq M_d$}{\Return $FindK(\mathbb M, \emptyset, k-|\mathbb N|)$\;}{\Return $FindK(\{M_i\in\mathbb M|i>d\}, \mathbb N, k-d)$\;}
    }
    \If{$M_d=N_d$}{
        \lIf{$k$为偶数}{\Return $M_d$(或$N_d$)}
        \lIf{$k$为奇数}{\Return $min(M_{d+1},N_{d+1})$}
    }
    \eIf{$M_d<N_d$}{\Return $FindK(\{M_i\in\mathbb M|i>d\}, \mathbb N, k-d)$\;}{\Return $FindK(\mathbb M, \{N_i\in\mathbb N|i>d\}, k-d)$\;}
}
\end{algorithm}
\begin{algorithm}[htbp]
    \caption{找两个正序数组的中位数}\label{alg:center}
    \SetKwProg{Fn}{Function}{ begin}{end}
    \Fn{FindMedium($\mathbb M$, $\mathbb N$)}{
        \KwIn{正序数组$\mathbb M$、$\mathbb N$}
        \KwOut{数组$\mathbb M$和$\mathbb N$的中位数}
        \eIf{$|\mathbb M|+|\mathbb N|$为奇数}{
            \lIf{$\mathbb M=\emptyset$}{\Return $N_{\frac{|\mathbb M|+|\mathbb N|+1}{2}}$}
            \lIf{$\mathbb N=\emptyset$}{\Return $M_{\frac{|\mathbb M|+|\mathbb N|+1}{2}}$}
            \Return $FindK(\mathbb M, \mathbb N, \frac{|\mathbb M|+|\mathbb N|+1}{2})$\;
        }{
            \lIf{$\mathbb M=\emptyset$}{\Return $(N_{\frac{|\mathbb M|+|\mathbb N|}{2}}+N_{\frac{|\mathbb M|+|\mathbb N|}{2}+1})/2$}
            \lIf{$\mathbb N=\emptyset$}{\Return $(M_{\frac{|\mathbb M|+|\mathbb N|}{2}}+M_{\frac{|\mathbb M|+|\mathbb N|}{2}+1})/2$}
            \Return $(FindK(\mathbb M, \mathbb N, \frac{|\mathbb M|+|\mathbb N|}{2})+FindK(\mathbb M, \mathbb N, \frac{|\mathbb M|+|\mathbb N|}{2}+1))/2$\;

        }
    }
    \end{algorithm}