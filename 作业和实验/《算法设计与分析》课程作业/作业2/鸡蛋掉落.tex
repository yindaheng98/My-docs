% !TeX root = ./homework.tex
\newpage\quad\newpage
\section{鸡蛋掉落}
\subsection{问题描述}
\subsubsection*{Description}
你将获得$K$个鸡蛋,并可以使用一栋从$1$到$N$共有$N$层楼的建筑。

每个蛋的功能都是一样的,如果一个蛋碎了,你就不能再把它掉下去。

你知道存在楼层$F$,满足 $0\leq F\leq N$任何从高于$F$的楼层落下的鸡蛋都会碎,从$F$楼层或比它低的楼层落下的鸡蛋都不会破。

每次移动,你可以取一个鸡蛋(如果你有完整的鸡蛋)并把它从任一楼层$X$扔下(满足$1\leq X\leq N$)。

你的目标是确切地知道$F$的值是多少。

无论$F$的初始值如何,你确定$F$的值的最小移动次数是多少?

\subsubsection*{Input}

第一行输入\code{nums}表示有\code{nums}组测试

每组测试输入\code{K},\code{N},表示有$K$个鸡蛋,$N$层楼

\subsubsection*{Output}

对每组测试数据,输出确定$F$的最小移动次数

\subsubsection*{Sample Input}

\code{3}

\code{1 2}

\code{2 6}

\code{3 14}

\subsubsection*{Sample Output}

\code{2}

\code{3}

\code{4}

\subsubsection*{提示}
1 <= \code{K} <= 1000

1 <= \code{N} <= 10000

\subsection{算法思路}

设满足条件的$F$最小移动次数为$M=f(K,N)$,令$n=g(k,m)$表示使用$k$个鸡蛋移动$m$次可以确定$F$的最大楼层数。对于此函数,可以推理出如下性质:
\begin{enumerate}
	\item 鸡蛋数量和移动次数至少为1,即:$$k>1,m>1$$
	\item 如果鸡蛋数量大于移动次数,则多出来的鸡蛋必定不会被用到,因此可以确定的楼层数和鸡蛋数量等于移动次数的情况相同,即:$$(\forall k<m)g(k,m)=g(m,m)$$
	\item 对于只有一个鸡蛋和一次移动的情况($k=m=1$),如果把鸡蛋从1楼扔下,若碎了,则确定$F=0$,若没碎,则无法确定$F$;但如果把它从较高楼层上扔下,不管鸡蛋碎没碎,都无法确定$F$,因此有这种情况最多只能确定$F=0$的情况,即:$$g(1,1)=0$$
	\item 进一步,对于有一个鸡蛋和$m$次移动的情况,显然只能从$1\sim m$楼依次扔鸡蛋,因为如果从较高层扔鸡蛋碎了,就无法确定$F$的准确值。这时最大的$F$出现于最后一次从$m$层扔下时鸡蛋碎了的情况,即$F=m-1$的情况,因此有:$$g(1,m)=m-1$$
	\item 更进一步,对于有两个鸡蛋和$m$次移动的情况,显然,可以一开始就将鸡蛋从第$m$层扔下,如果碎了,说明$F\leq m-1$,剩下的$m-1$个鸡蛋可以保证在$m-1$层之内找到$F$,可以确定$F$的楼层数最高为$F=m-1$层;如果没碎,说明$F\geq m$,那么剩下的两个鸡蛋和$m-1$次就是一个从$m$层开始的$g(2,m-1)$问题,即最大可确定$F$的层数为:$$g(2,m)=m+g(2,m-1)$$
	\item 同理可知,对于$k$个鸡蛋和$m$次移动的情况,可以一开始就将鸡蛋从第$g(k-1,m-1)+1$层扔下,如果碎了,剩下的$k-1$个鸡蛋和$m-1$次移动可以保证在$g(k-1,m-1)$层内找到$F$;如果没碎,那么剩下的$k$个鸡蛋和$m-1$次移动就是一个从$g(k-1,m-1)+1$层开始的$g(k,m-1)$问题,即:$$g(k,m)=g(k-1,m-1)+1+g(k,m-1)$$
\end{enumerate}

综上所述,求解$N=g(K,M)$可以使用递归,按照$g(k,m)=g(k-1,m-1)+1+g(k,m-1)$进行递归计算即可。进而,对$M=f(K,N)$的求解可以转化为一系列$N=g(K,m)$的问题,即令$m$从1开始,依次计算$n=g(K,m)$直到$n=g(K,m)\geq N$,此时的$m$即为所求值。

\subsection{算法伪代码}

见算法\ref{alg:s2}。
\begin{algorithm}[htbp]
\caption{鸡蛋掉落算法伪代码}\label{alg:s2}
\SetKwProg{Fn}{Function}{ begin}{end}
\Fn{f($K$, $N$)}{
    \KwIn{鸡蛋个数$K$、楼层高度$N$}
	\KwOut{确定$F$的值的最小移动次数$M$}
	\Fn{g($k$,$m$)}{
		\KwIn{鸡蛋个数$k$、移动次数$m$}
		\KwOut{$k$个鸡蛋移动$m$次可以确定$F$的最大楼层数$n$}
		\lIf{$k=1\vee m=1$}{\Return{$1$}}
		\Return{$g(k,m)=g(k-1,m-1)+1+g(k,m-1)$}
	}
	$m=1$\;
	\For{$m\in\{m|g(k,m)<N\}$}{
		$m=m+1$\;
	}
	\Return{$m$}
}
\end{algorithm}