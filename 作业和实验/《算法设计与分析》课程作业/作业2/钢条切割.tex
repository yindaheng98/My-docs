% !TeX root = ./homework.tex
\section{钢条切割}
\subsection{问题描述}
\subsubsection*{Description}

给定一根长度为$n(n\leq10000)$的钢条以及一张价格表,请计算这根钢条能卖出的最大总收益。价格表表示为$(l_i,p_i), 1\leq i\leq k$。不在价格表中的钢条可卖出价格为$0$。

\subsubsection*{Input}

第一行输入\code{m(m$\leq$10)}表示有M组数据。每组数据第一行输入两个\code{int}型整数\code{n}和\code{k},分别表示钢条长度以及价格表中不同价格数量。接下来一行输入\code{k}个价格的表示$(l_i,p_i)$,均为整数,$l_i$可能大于\code{n}。

\subsubsection*{Output}

输出\code{m}行整数,第\code{i}行表示第\code{i}组数据的最大总收益。

\subsubsection*{Sample Input}

\code{2}

\code{27 3}

\code{35 41 61 49 73 74}

\code{94 2}

\code{21 55 88 64}

\subsubsection*{Sample Output}

\code{0}

\code{220}

\subsection{算法思路}

令$S_{L}=\{L_i|i\in[1,N]\}$表示长$L$的钢条在价格表$P=\{(l,p_l)|l\in\mathbb{N},p\in\mathbb{R}\}$下的最优切割方案,其价格为$P(S)$,显然最优切割方案有如下性质:

\begin{enumerate}
    \item 最优子结构:$$(\forall S\subseteq S_{L})S=S_{\sum_{l\in S}l}$$
    \item 重叠子问题:$$P(S_L)=max\{P(S_{L-l})+p_l|(l,p_l)\in P\}$$
\end{enumerate}

因此可以采用查表法,求解长$L$的钢条的最优切割方案时,使用数组存储$P(S_i)$,从$i=0$开始依次计算$P(S_L)=max\{P(S_{i-l})+p_l|(l,p_l)\in P\}$直到$i=L$即为要求的$P(S_L)$。

\subsection{算法伪代码}
见算法\ref{alg:cut}。
\begin{algorithm}[htbp]
\caption{钢条切割算法伪代码}\label{alg:cut}
\SetKwProg{Fn}{Function}{ begin}{end}
\Fn{SteelCut($L$, $P$)}{
    \KwIn{钢条长度$L\in\mathbb{N}_+$、价格表$P=\{(l,p_l)|l\in\mathbb{N},p\in\mathbb{R}\}$}
    \KwOut{最佳切割方案价格$P(S_L)$}
    \For{$i\in\{1,2,3,...,L\}$}{
        $P_i=0$\;
        \For{$(l,p_l)\in P$}{
            \eIf{$l\leq i$}{
                $P_i=max(P_i,l+P_{i-l})$\;
            }{
                break\;
            }
        }
        记下$P_i$\;
    }
    \Return $P_L$
}
\end{algorithm}