% !TeX root = ./homework.tex
\section{最长公共子序列}
\subsection{问题描述}
\subsubsection*{Description}
给定两个字符串$A$和$B$,请计算这两个字符串的最长公共子序列长度。

\subsubsection*{Input}

第一行输入\code{M(M$\leq$10)}表示有\code{M}组数据。每组数据输入两行字符串,字符串的长度不长于500。

\subsubsection*{Output}

输出\code{M}行正整数,第\code{i}行表示第\code{i}组数据的最长公共子序列长度。

\subsubsection*{Sample Input}

\code{2}

\code{abcdefg}

\code{cemg}

\code{abcdefgh}

\code{ceaaegh}

\subsubsection*{Sample Output}

\code{3}

\code{4}

\subsection{算法思路}

由于题干并没有要求找任意第$k$大的数,因此本题可以直接使用线性查找,时间复杂度为$O(n)$。

\subsection{算法伪代码}
见算法\ref{alg:fn}。
\begin{algorithm}[htbp]
\caption{找第二大数算法伪代码}\label{alg:fn}
\SetKwProg{Fn}{Function}{ begin}{end}
\Fn{FindNumber($S$, $n$)}{
    \KwIn{数组$S$、数组长度$n$}
    \KwOut{数组$S$中第二大的数}
    $
    R=\left\{
        \begin{aligned}
            \{S_0, S_1\}&(S_0<S_1)\\
            \{S_1, S_0\}&(S_0>S_1)\\
        \end{aligned}
    \right.
    $\;
    \For{$i\in\{2,3,...,n\}$}{
        \If{$S_i<R_0$}{
            $R_1=R_0$\;
            $R_0=S_i$\;
        }\Else{
            $R_1=S_i$\;
        }
    }
    \Return $R_1$\;
}
\end{algorithm}