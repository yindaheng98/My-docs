% !TeX root = ./homework.tex
\section{题解2}
\subsection{问题分析}
最长公共子串问题是一个典型的动态规划问题,最长公共子序列在删去最后一个公共字符后,仍然是剩下字符串的最长公共子序列,因此可得如下递推公式:
$$
C[i,j]=\left\{
\begin{aligned}
    max\{C[i-1,j],C[i,j-1]\}&\quad&(A[i]=B[j])\\
    C[i-1,j-1]+1\}&&(A[i]\not =B[j])\\
\end{aligned}
\right.
$$

\subsection{算法伪代码}
见算法\ref{alg:2}。
\begin{algorithm}[htbp]
\caption{题解2算法伪代码}\label{alg:2}
\SetKwProg{Fn}{Function}{ begin}{end}
\Fn{F($S$)}{
    \KwIn{字符串$A,B$}
    \KwOut{最长公共子串长度$C$}
    \leIf{$A[i]=B[j]$}{$C[1,1]=1$}{$C[1,1]=0$}
    \For{$i\in[2,m]$}{
        \For{$j\in[2,n]$}{
            $$
            C[i,j]=\left\{
            \begin{aligned}
                max\{C[i-1,j],C[i,j-1]\}&\quad&(A[i]=B[j])\\
                C[i-1,j-1]+1\}&&(A[i]\not =B[j])\\
            \end{aligned}
            \right.
            $$
        }
    }
    \Return $C[m,n]$\;
}
\end{algorithm}