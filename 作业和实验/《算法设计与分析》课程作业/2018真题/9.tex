% !TeX root = ./homework.tex
\section{题解9}
\subsection{问题分析}
显然,问题的解空间是,每一张骨牌可以放的位置都只有有限的几个,因此,解空间$(x_1,x_2,x_3,\dots,x_28)$就是每个骨牌摆放的位置,搜索树的每一层节点表示一个骨牌的可选摆放位置,搜索树以根节点为第$0$层,第$i$层的分支就是选择第$i+1$号骨牌的位置。

\subsection{剪枝策略}
\begin{itemize}
    \item 选择分支时,如果要放的位置以被其他骨牌占据,显然可以剪掉
    \item 选择分支时,如果放置后周围的格子中有孤立的格子,显然可以剪掉
\end{itemize}