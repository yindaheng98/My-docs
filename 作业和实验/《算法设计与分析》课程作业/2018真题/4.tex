% !TeX root = ./homework.tex
\section{题解4}
\subsection{问题分析}
显然,本题的贪心选择性是:每个小朋友$p_i$都只要选择能与自己匹配的最大的$p_j$即可。

可以通过反证法证明此方法最优:如果对于某个配对$p_i,p_j$,存在一个$p_k<p_j$,使$p_i$不与$p_j$配对而与$p_k$配对时,能增加一个配对$p_j,p_l$,如果要这种情况成立,那么必然有:
$$
\left\{
\begin{aligned}
    p_i+p_j\leq B\\
    p_k<p_j\\
    p_i+p_k\leq B\\
    p_k+p_l>B\\
    p_j+p_l\leq B\\
\end{aligned}
\right.
$$
但显然,$p_j+p_l>p_k+p_l>B$,不可能有$p_j+p_l\leq B$,因此此方法不是最优的情况不存在。

\subsection{算法伪代码}
见算法\ref{alg:4}。
\begin{algorithm}[htbp]
\caption{题解4算法伪代码}\label{alg:4}
\SetKwProg{Fn}{Function}{ begin}{end}
\Fn{F($p_i$)}{
    \KwIn{小朋友的数字牌$\{p_i\}$、匹配和限制$B$}
    \KwOut{能匹配的最多对小朋友$C$}
    对$\{p_i\}$进行排序\;
    \Repeat{$\{p_i\}$中只剩1个或0个元素}{
        选出$\{p_i\}$中最大的元素$p$\;
        从$\{p_i\}$中删去$p$\;
        \lIf{$p\geq B$}{continue}
        选出$\{p_i\}$中$\leq B-p$最大的元素$p'$\;
        从$\{p_i\}$中删去$p'$\;
        $C=C+1$\;
    }
    \Return{$C$}
}
\end{algorithm}