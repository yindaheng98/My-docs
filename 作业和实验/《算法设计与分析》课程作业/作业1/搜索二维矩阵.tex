% !TeX root = ./homework.tex
\section{搜索二维矩阵}
\subsection{问题描述}
\subsubsection*{Description}
编写一个高效的算法来搜索\code{m x n}矩阵\code{matrix}中的一个目标值\code{target}。该矩阵具有以下特性:
\begin{enumerate}
	\item 每行的元素从左到右升序排列。
	\item 每列的元素从上到下升序排列。
\end{enumerate}

\subsubsection*{Input}

第一行输入\code{nums}表示有\code{nums}组测试。

每组测试输入\code{m}、\code{n},\code{target},分别表示矩阵的行列数以及目标值。

接下来输入\code{m * n}的二维矩阵。

\subsubsection*{Output}

对每组测试数据输出 能否在矩阵中找到target。

若能找到,输出\code{true}。

若找不到,输出\code{false}。

\subsubsection*{Sample Input}

\code{1}

\code{5 5 5}

\code{1 4 7 11 15}

\code{2 5 8 12 19}

\code{3 6 9 16 22}

\code{10 13 14 17 24}

\code{18 21 23 26 30}

\subsubsection*{Sample Output}

\code{true}

\subsubsection*{提示}
\code{m} <= 1000

\code{n} <= 1000

\subsection{算法思路}

设目标值为$x$,要查找的矩阵为$A_{m\times n}=(a_{i,j})$,满足:

$$
\left\{
\begin{aligned}
a_{i,j}\leq a_{i,j+1}&\quad (i\in[1,m],j\in[1,n-1])\\
a_{i,j}\leq a_{i+1,j}&\quad (i\in[1,m-1],j\in[1,n])\\
\end{aligned}
\right.
$$

显然有如下定理:

$$
\begin{aligned}
x>a_{0,n}\Rightarrow(\forall a_{1,j},j\in[1,n])(x>a_{0,j})\\
x<a_{0,n}\Rightarrow(\forall a_{i,n},i\in[1,m])(x>a_{i,n})\\
\end{aligned}
$$

即当目标值大于矩阵右上角值时,目标值必大于矩阵第一行的值;目标值小于矩阵右上角值时,目标值必小于矩阵第一列的值。因此,我们可以从矩阵右上角开始搜索,若目标值较大,则向下搜索;若目标值较小,则向左搜索,直到找到目标值或超出矩阵范围。此算法时间复杂度为$O(m+n)$。

更进一步,对于向下或向左的搜索过程,我们可以使用二分查找,找出当前行不大于目标值的最大元素(目标值较小时)或当前列不小于目标值的最小元素(目标值较大时)。改进后算法的时间复杂度可以达到$O(log(m)+log(n))$。

\subsection{算法伪代码}

见算法\ref{alg:s2}。
\begin{algorithm}[htbp]
\caption{搜索二维矩阵算法伪代码}\label{alg:s2}
\SetKwProg{Fn}{Function}{ begin}{end}
\Fn{FindNumber($A_{m,n}$, $x$)}{
    \KwIn{待查矩阵$A_{m,n}=(a_{i,j})$、目标值$x$}
	\KwOut{$A_{m,n}$中是否存在$a_{i,j}=x$}
	\lIf{$m=0\vee n=0$}{\Return false}
	\lIf{$x=a_{1,n}$}{\Return true}
	\eIf{$x>a_{1,n}$}{
		二分查找满足条件的$k$:$k>1\wedge a_{k,n}\geq x\wedge a_{k-1,n}< x$\;
		\Return $FindNumber(A'_{k,n}=(a_{i,j})(i\in[k,n],j\in[1,n]),x)$\;
	}{
		二分查找满足条件的$k$:$k>1\wedge a_{1,k}\leq x\wedge a_{1,k+1}> x$\;
		\Return $FindNumber(A'_{m,k}=(a_{i,j})(i\in[1,n],j\in[1,k]),x)$\;
	}
}
\end{algorithm}