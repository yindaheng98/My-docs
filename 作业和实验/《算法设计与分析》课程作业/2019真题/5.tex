% !TeX root = ./homework.tex
\section{题解5}
\subsection{问题分析}
此问题可以通过分治求解,对于上部的焊点$T_i$和下部的焊点$B_i$,可以通过如下方式求它们的相交次数:
\begin{itemize}
    \item 分治:分别递归地计算$T_i,i\in[1,n/2]$和$T_i,i\in[n/2+1]$的相交点个数,并对$T_i,i\in[1,n/2]$和$T_i,i\in[n/2+1]$所连接的下部焊点编号进行排序
    \item 组合:对于$T_i,i\in[1,n/2]$,将其对应的上部焊点编号与$T_i,i\in[n/2+1]$对应的上部焊点编号比较,$T_i,i\in[1,n/2]$上部编号较大的与$T_i,i\in[n/2+1]$上部编号较小的相交,求出相交数量;对于$T_i,i\in[n/2+1]$同理。最终求出相交点数之和即最终结果。
\end{itemize}

\subsection{算法伪代码}
见算法\ref{alg:5}。
\begin{algorithm}[htbp]
\caption{题解5算法伪代码}\label{alg:5}
\SetKwProg{Fn}{Function}{ begin}{end}
\Fn{$F(T)$}{
    \KwIn{上部焊点对应的下部焊点编号$T$,$T_i=j$表示上部焊点$i$与下部焊点$j$相连}
    \KwOut{相交焊点数$S$,所有的下部焊点编号排序结果$B$}
    $S_1,B_1=F(\{T_i|i\in[1,|T|/2]\})$\;
    $S_2,B_2=F(\{T_i|i\in[|T|/2+1,|T|]\})$\;
    $S=S_1+S_2$\;
    $B=\{\},k=1,t=0$\;
    \Repeat{}{
        \eIf{$B_1[i]>B_2[j]$}{
            $B[k]=B_2[j]$\;
            $j=j+1$\;
            $t=t+1$\;
        }{
            $B[k]=B_1[i]$\;
            $i=i+1$\;
            $S=S+t$\;
            $t=j-1$\;
        }
        $k=k+1$\;
    }
    \Return $S,B$
}
\end{algorithm}