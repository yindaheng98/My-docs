% !TeX root = ./homework.tex
\section{题解6}
\subsection{问题分析}
显然,对于$n$个点,它们的间距有$n-1$个,将其组织为一个数组$A$,原覆盖问题就可以看作选择这个数组中的连续几个值,显然,总共有$(n-1)+(n-2)+\dots+2+1=n(n-1)/2$种选法遍历所有这些选法就能得到$O(n^2)$时间的算法。

但实际上这$n(n-1)/2$种选法并不需要全部遍历,可以从第一个间距开始增加,如果长度超过绳长度,就将开头处的距离减去,直到增加到最后一个间距,时间复杂度为$O(n)$。

\subsection{算法伪代码}
$O(n^2)$时间的算法见算法\ref{alg:6-1}。
\begin{algorithm}[htbp]
\caption{题解6算法伪代码}\label{alg:6-1}
\SetKwProg{Fn}{Function}{ begin}{end}
\Fn{$F(T)$}{
    \KwIn{点的间距列表$A$,绳子长度$L$}
    \KwOut{覆盖的最大点数$m$}
    $m=0$\;
    \For{$i\in [1,|A|]$}{
        $S=0,k=1$\;
        \For{$j\in [i,|A|]$}{
            $S=S+A[j]$\;
            $k=k+1$\;
            \If{$S<=L\wedge k>m$}{
                $m=k$\;
            }
        }
    }
    \Return $m$
}
\end{algorithm}
$O(n)$时间的算法见算法\ref{alg:6-2}。
\begin{algorithm}[htbp]
\caption{题解6算法伪代码}\label{alg:6-2}
\SetKwProg{Fn}{Function}{ begin}{end}
\Fn{$F(T)$}{
    \KwIn{点的间距列表$A$,绳子长度$L$}
    \KwOut{覆盖的最大点数$m$}
    $m=0$\;
    \For{$i\in [1,|A|]$}{
        $S=0,k=1$\;
        \For{$j\in [i,|A|]$}{
            $S=S+A[j]$\;
            $k=k+1$\;
            \If{$S>L$}{
                break\;
            }
            \If{$S<=L\wedge k>m$}{
                $m=k$\;
            }
        }
    }
    \Return $m$
}
\end{algorithm}