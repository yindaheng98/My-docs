% !TeX root = ./homework.tex
\section{题解4}
\subsection{问题分析}
首先,该方法不是最优方法。如果$d_{12}=2$、$d_{24}=1$、$d_{34}=2$,且其余路径长度都满足$d_{ij}=d_{ik}+d{kj}$,即点$v_1,v_2,v_4,v_3$一字排开,那么算法首先会将$a_2$分给$r_1$,之后会将$a_1$分给$r_2$,代价总和为$6$,然而最优方案应该是$a_1$分给$r_1$、$a_2$分给$r_2$,代价总和为4。

该问题是一个典型的二部图最大匹配问题:
\begin{itemize}
    \item 构建二部图:计算每个出租车到每个乘客之间的距离,作为二部图的边权值
    \item 计算二部图:计算二部图的最小匹配方案
\end{itemize}

\subsection{算法伪代码}
见算法\ref{alg:4}。
\begin{algorithm}[htbp]
\caption{题解4算法伪代码}\label{alg:4}
\SetKwProg{Fn}{Function}{ begin}{end}
\Fn{QuickSort($A$)}{
}
\end{algorithm}